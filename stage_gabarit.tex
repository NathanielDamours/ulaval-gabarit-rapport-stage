\documentclass[12pt]{article} 	% Type de document & taille de police
\usepackage[margin=2.53cm]{geometry}  % Marges normales de Word
\usepackage[french]{babel}  % Langue
\usepackage[utf8]{inputenc}  % Saisi du texte : permet les accents
\usepackage[T1]{fontenc} 	% Encodage du document
\setlength{\parindent}{0pt}  % Pas d'indentation
\setlength{\parskip}{1em}  % Passer une ligne après chaque paragraphe
\usepackage{hyperref}	% Inclure les liens
\usepackage{graphicx}	% Inclure des images
\topskip0pt  % Permet \vspace*{\fill} pour centrer verticalement
\usepackage{xcolor} % Texte coloré

\begin{document}

\thispagestyle{empty}
		\begin{minipage}[t]{8.5cm}
			\vspace{0pt}
			\begin{flushleft}
				\hspace{-1cm}\includegraphics[width=5cm]{img/ulaval_logo.jpg}\\
				\hspace{-1cm}{\sf\scriptsize \textbf{FACULTÉ DES SCIENCES ET GÉNIE}}\\[-0.2cm]

			\end{flushleft}
		\end{minipage}
		\begin{minipage}[t]{8.5cm}
				\begin{flushright}		
						\hspace*{2cm} \\
						\hspace*{1cm}TITRE DU COURS \\
						\hspace*{1cm}\# DU COURS DE STAGE\\
						\hspace*{1cm}SESSION (ex : Été 2014)\\
						\hspace*{1cm}Baccalauréat en XXX\\
				\end{flushright}		
		\end{minipage}

\vspace{4cm}
\begin{center}
	\large TITRE DU STAGIAIRE - NOM DE LA COMPAGNIE \\
	\vspace{1cm}
	\large TYPE DE RAPPORT (ex : Rapport de mi-session ou Rapport d'étape ou Rapport de fin de stage...)\\

	\vspace{5cm}
	\fontsize{15}{15}\textbf {Destinataire}\\

	\vspace{0.5cm}
	\large Département des stages en milieu pratique \\

\end{center}
\vspace{3cm}
\begin{flushleft}
	Date de remise : DATE DE REMISE (ex : 12 mai 2014)
\end{flushleft}

\par\noindent\rule{\textwidth}{0.4pt}
\begin{minipage}[t]{6cm}
	\begin{flushleft}
		\textbf {PRENOM NOM}\\
		IDUL (ex : 123 456 789)
	\end{flushleft}
\end{minipage}
		\begin{minipage}[t]{10cm}
				\begin{flushright}		
						\hspace*{1cm}\textbf {NOM SUPERVISEUR - TITRE} \\
				\end{flushright}		
		\end{minipage}

\newpage
\pagenumbering{Roman}
\textbf{Résumé}  % (moins de 1⁄2 page)
% Présent et passé composé
% Un résumé doit présenter : le but et la nature du travail, les méthodologies utilisées, les principaux résultats et les principales conclusions. Que devrait savoir quelqu’un qui n’a pas le temps de lire tout votre rapport? Quel en est l’essentiel?
% Exemple : « Ce rapport présente le travail effectué par (nom) dans le cadre du stage de formation en entreprise à la compagnie (...) pendant la période (du ...), dans le département (de ...) et qui consistait à .... »


\newpage
% Remerciements
\vspace*{\fill}
	\begin{center}	
		JE TIENS À REMERCIER...
	\end{center}	
\vspace*{\fill}


\newpage
% Table des matières
\tableofcontents


\newpage
% Listes des tableaux
\listoftables


\newpage
% Table des figures
\listoffigures


\newpage
\textbf{Liste de symboles et abréviations}\\
\break


\newpage
\pagenumbering{arabic}

\section{Introduction}  % Max 1 page
% Problématique -- > Présent
% Hypothèse (s’il y a lieu) -- > Conditionnel
% Mandat -- > Présent
% Contenu -- > Présent

\subsection{Présentation personnelle}
% Présentation personnelle: nom, programme, cheminement scolaire et professionnel dont l’avancement dans le programme et les stages

\subsection{Présentation de l’organisation}
% Nom de l’entreprise ou du centre de recherche, secteur d’activité, mission, taille, services / produits, effectifs, localisation

\subsection{Présentation du stage et de son environnement}
% Présentation du/des projet(s) dans le cadre de la mission de l’organisation ou du domaine de recherche, titre du stagiaire, environnement de travail, taille de l’équipe, Période du stage, type d’encadrement : superviseur (titre et fonction), lieu de travail, etc.


\newpage
\section{Responsabilités et tâches du stagiaire}  %  Présent
% Le contenu principal du rapport qui décrit toutes les étapes franchies et les moyens mis de l’avant pour solutionner les différentes problématiques, une validation des résultats obtenus et la formulation de recommandations pour le futur.
% Comment la formation en milieu pratique sera porteuse pour la suite des études de baccalauréat ou pour la fin de la formation universitaire.

\subsection{Rôle et contribution du stagiaire}

\subsection{Objectifs, problématique, méthodologie, théorie dans le cadre d’un stage en recherche}  % Méthodologie : passé composé
% Contraintes, moyens disponibles, échéancier, etc.

\subsection{Description des tâches et des travaux effectués}
% Description des tâches et des travaux effectués

\subsubsection{Mandat : TITRE DU MANDAT}

\subsubsection{Mandat : TITRE DU MANDAT}
% Ajoutez des mandats si nécessaires.

\subsection{Résultats / analyses et discussions}

\subsubsection{Mandat : TITRE DU MANDAT} 

\subsubsection{Mandat : TITRE DU MANDAT}
% Ajoutez des mandats si nécessaires.

\subsection{Comparaison avec les attentes du stagiaire avant le début du stage}
% Comparaison avec les attentes du stagiaire avant le début du stage en cas de premier stage.
% Répondre à ceux applicables.

\newpage
\section{Développement et renforcement des compétences}

\subsection{Techniques} 
% Terrain, laboratoire, programmation, etc.

\subsection{En ingénierie ou scientifiques}
% Recherche, conception, développement, analyse, suivi, gestion de projet, etc.

\subsection{Communication}
% Rapports, présentation orale, interactions avec différents types d’intervenants, etc.

\subsection{Réflexion sur la formation pratique et théorique reçue} % Présent
% Pratique : description d’une méthode de travail acquise ou adoptée (exemple : en développement, en analyse, réunions, etc.) durant le stage ;

% Organisation du travail adoptée durant le stage : gestion du temps, respect des échéances, gestion des priorités, planification des tâches.

% Professionnalisme : éthique, contrôle de qualité, santé et sécurité, protection de l’environnement, etc.) Bilan sur l’atteinte des objectifs individuels et de l’employeur fixés en mi-stage. Commentaires sur la recherche de stage. La formation pré stage. Les ajustements personnels et académiques possibles. Comment votre stage vous permet de mieux cibler le type de carrière que vous envisagez.

% Théorique : une question à poser à un enseignant après le stage dans un cours suivi ou à suivre. Un cours à option que le stagiaire choisirait après le stage. Une recommandation que le stagiaire ferait au directeur de son programme.

\subsection{Bilan des acquis}
% En 3-4 lignes, bilan des acquis (par rapport vos attentes avant le stage, les points forts et faibles, vos apprentissages en lien avec votre formation ou vos méthodes de travail, etc.).
% Avenir professionnel et/ou académique du stagiaire.


\newpage
\section{Conclusion}

\subsection{Rétrospective}
% Sur les principales contributions, réalisations, état d’avancement du travail ou projet décrit dans le rapport. Remise en contexte du travail et des objectifs, ouverture possible vers d’autres contextes, améliorations, etc. 

\subsection{Perspectives pour l’avenir} 
% Académique ou marché du travail? Réflexion sur l’intérêt pour ce type de mandat après un stage. Recommandations, etc.


\section{TUTORIEL POUR LA BIBLIOGRAPHIE}

%%%%%%%%%%%%%%%%%%%%%%%%%%%%%%%%%
\textbf{
	\textcolor{red}{
		Enlevez les sections TUTORIEL avant la remise de votre rapport
						   }	
           }
%%%%%%%%%%%%%%%%%%%%%%%%%%%%%%%%%

Ajoutez vos sources au document \textbf{references.bib}.

\textbf{Citer une source} : \textbackslash cite\{SOURCE\} pour obtenir ceci : \cite{howard2020deep}.

\textbf{Citer plusieurs sources} : \textbackslash cite\{SOURCE1, SOURCE2\} pour obtenir ceci : \cite{goodfellow2014generative, CycleGAN2017}.

\textbf{Citer une figure} : \textbackslash ref\{fig:NOM\}. Par exemple, voir la Figure~\ref{fig:gan} ou voir la Figure~\ref{fig:fastbook} en annexe.

\begin{figure}[h!]
	\begin{center}
		\includegraphics[scale=0.4]{img/gan_img.jpg}
	\end{center}
  \caption{Figure dans le texte~\cite{CycleGAN2017}}
  \label{fig:gan}
\end{figure}

\newpage
\appendix

\section{TUTORIEL (SUITE)}

\begin{figure}[h!]
	\begin{center}
		\includegraphics[scale=1.3]{img/fastbook.png}
	\end{center}
  \caption{NON VALIDES, VOIR COMMENTAIRES\cite{howard2020deep}}  % « Tu dois faire figurer le lien directement dans la légende de l'image. » (Adjoint à la coordination des stages), ce que je ne suis pas arrivé à faire.
  \label{fig:fastbook}
\end{figure}


\newpage
\bibliographystyle{IEEEtran}  % « La bibliographie est la seule chose que vous pouvez structurer en anglais notamment les références IEEE qui ont une rigueur stricte. » (Adjoint à la coordination des stages)
\bibliography{references}

\end{document}
